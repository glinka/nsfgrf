I've lived a busy life. I'm told it started at birth, when I gasped my first breaths in, what I imagine was, a state of terrible panic at my new surroundings. I do, however, have vague recollections of Kindergarten, where I was set to work on the basics of handwriting, shoe-tying and friend-making. Friend-making remained my chief occupation throughout elementary and middle school, as I anxiously awaited phonecalls inviting me to birthdays or sleepovers. In high school, blessed with a decent intellect, I 

I find myself increasingly confronted by the question: ``What am I doing?'' Not the ``this six-month-old handlebar mustache has not lived up to expectations'' sort of  ``what am I doing?'' but rather the more philosophical kind that follows the consideration ``time to take stock of my life's direction''. This question appeared unanswered at the start of this past summer, when I bid farewell to my seventeen-year-old occupation of being a full-time student. I imagine I had previously considered it, but, before, it was always easily dismissed with ``taking classes'' or ``getting good grades'' or ``making friends''. Even now I could say ``researching towards a doctorate'', but I find this response incomplete. Thus, to speak candidly, my previous application's personal statement, written a year ago, was insincere. I had the obligatory list of community service and mission trips intending to show my deep concern for humanity, but in hindsight, these were more shortsighted answers to ``what am I doing?'': now I work at a soup kitchen, now I teach disadvantaged youth, and so on. I feel incredibly blessed that, as I begin to more fully grapple with this question, I find myself in my current position: researching complex networks at Princeton University. Because, while I may have stumbled onto this path, I see that it may lead to real solutions to serious problems. Standing here, and understanding the importance of the question before me, I'm aware for the first time that the choice of direction is mine.
