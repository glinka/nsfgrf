\documentclass[11pt]{article}
%% \setlength{\textheight}{210mm}
%% \addtolength{\topmargin}{-15mm}
%% \setlength{\textwidth}{155mm}
%% \setlength{\oddsidemargin}{5mm}
\usepackage{graphicx, subcaption, amsfonts}
\usepackage[top=1in, bottom=1in, left=1in, right=1in]{geometry}
\graphicspath{ {./figs/} }
\pagestyle{plain}
\begin{document}
I find myself increasingly confronted by the question: ``what am I doing?'' Not the ``this six-month-old handlebar mustache has not lived up to expectations'' sort **too cute** of  ``what am I doing?'' but rather the more philosophical kind that follows the consideration ``time to take stock of my life's direction''. This question appeared unanswered at the start of this past summer, when I bid farewell to my previous occupation, held from the age of five, of being a full-time student. Not that it was an entirely new consideration, but, before, it was always easily dismissed with ``taking classes'' or ``getting good grades'' or ``making friends''. Even now I could say ``researching towards a doctorate'', but I find this response incomplete. Thus, to speak candidly, my previous application's personal statement, written a year ago, was disingenuous. I had the obligatory list of community service and mission trips intending to show my deep concern for humanity, but in hindsight, these were more shortsighted answers to ``what am I doing?''. As I begin to more fully grapple with this question, I feel incredibly blessed to be in my current position: researching complex networks at Princeton University. Because, while I may have stumbled onto this path, I see that it may lead to real solutions to serious problems. Standing at this junction, and understanding the importance of the question before me, I can no longer afford to blindly follow life's trajectory, and am aware for the first time that the choice of direction into the future is mine. Admittedly, the way forward is unclear. But I have personally determined that it must be found by considering its potential impact on humanity above all else.\\

This recent search is largely a result of my inability to justify my current position in life. I recently watched a documentary that displayed, in graphic detail, the effect of the Syrian civil war on the country's average citizens. What started as a calm day in the small village of al Bara quickly devolved into a scene I still cannot fully absorb. Within seconds of spotting military jets overhead, what had been an entire block of homes and businesses and life was reduced to uneven rubble. A mother screamed for help as her son lay unconsious, trapped under a pile of stones while others were rushed off in makeshift ambulances for what little medical care was available. The planes made another pass, dispersing the crowd that had gathered to help as everyone ran for cover, but the target was a different section of the town, which was levelled with the same immediacy. And I have the audacity to complain about proposal deadlines! I know I cannot fathom the daily sorrows of so many millions in the world who were not born in suburban America, but I know, too, that complacency is not an option. I am convinced that I must use the incredible opportunities afforded me to work towards some sort of good for those less fortunate.\\

I plan to use my time in graduate school to build a foundation for this future work. Studying complex networks, the range of research is wonderfully broad, granting flexibility in application areas. Thankfully, there can be significant overlap in the analysis of completely different network problems. As the research proposal explains, our current goal is to develop additional analytical techniques that are broadly applicable across the field. This motivates me, as I view each advancement as a building block to future, targeted applications. Already I've learned a tremendous amount. Simulating network dynamics has given me thorough understanding of the C++ and Python programming languages. A course I took during my first year of graduate school taught me the many methods of dimensionality reduction used with massive dataset; another introduced the basics of nonlinear dynamics. I'm currently studying probability theory which will be invaluable in modeling and understanding the stochastic network systems so prevalent in real life. All of this beautfifully serves the dual purposes of advancing my current research aims and developing a general set of tools for future work in many different disciplines. This says little to my ability, however, to actually perform meaningful research. As such, my past and current research experiences are elaborated below.\\

My first taste of academic research followed an award for an NSF grant for “International Research and Education in Engineering”, allowing me to spend my freshman summer studying at Shanghai Jiao Tong University. The project, in their Institute of Refrigeration and Cryogenics, looked at computational fluid dynamic (CFD) simulations of a solar-powered air conditioning system. A panel of solar cells was used to cool water, which was then pumped through a grid of thin pipes on the ceiling of a room. This method of cooling using a mesh of pipes was yet unproven. If it proved feasible, there were several variables that required optimization, namely circulation-fan placement and speed, and mesh density. “ANSYS Airpak” CFD software was used first to validate the approach, and then to quickly screen for suitable configurations.\\


Working in China, I faced a significant communication barrier. The Chinese graduate students spoke broken English, and my Mandarin was limited to essential everyday phrases such as “Where is the bathroom” and “I know full well that wallet isn’t made of real leather, and will not pay full price for it.” Thus I was forced into mostly independent study. This proved a challenge as, having only one year of college and no prior experience in CFD simulations, I faced a significant learning curve. After poring over the software manual and completing a number of online tutorials, I learned some basic fluid dynamics, and had a grasp of how to effectively use the software. After showing that the grid-cooling system was feasible, I went on to suggest a set of optimal room parameters. These results were given in a final report, presented to the head of the group, Prof. Yong Li, at the end of the summer. Additionally, a final report detailing my recommendations was submitted in fulfillment of the grant.\\

The following summer, I worked in Cargill’s “Global Food Research” division as a research intern. The project, this time purely experimental, was to establish a method for producing a specific carbohydrate gel for saturated fat replacement in food. Unfortunately I cannot go into specific details given the commercial aspects of the project, but I was again working independently, creating novel synthesis and analytical procedures based on the work of previous researchers and existing literature. During the course of my investigation, I was able to employ chemical engineering fundamentals from mass and heat transfer to show that certain synthesis procedures we had been attempting were thermodynamically impossible. My findings were presented to a committee of senior researchers and managers. Due, in part, to my suggestions, management deemed the project unfeasible, and invested their resources elsewhere.\\

Following my work at Cargill, I sought out research in computational chemical engineering. This led me to the lab of Professor Renata Wentzcovitch. It was in her group that I completed research for my senior thesis, in the area of stress-strain relations in crystal systems. The stress-strain relation of a crystal is a three dimensional generalization of Hooke’s Law, which, in Einstein tensor notation, reads
\[
\sigma_{ijk}=C_{ijkl}e_{kl}  \ \ i,j,k = 1,2,3
\]
Thankfully, the 81 elements of the tensor C (the elastic tensor), can be reduced through thermodynamic arguments to 21, and the resulting relationship turned into the more manageable
\[
\sigma_{i}=C_{ij}\epsilon_{j} \ \ i,j=1,2,3,4,5,6
\]

This form lends itself to easy matrix manipulation. The problem at hand was the calculation of the coefficients of the elastic tensor from arbitrarily many pairs of pre-calculated stress and strain vectors. In general, this would be a straightforward calculation, but in this specific problem, the elastic tensor is often singular, depending on the symmetry elements of the crystal under investigation. Previous programs had taken a case-by-case approach to the problem, in which explicit formulas were used to solve the elastic tensor in a given symmetry. The program I developed was a general solver, and the same algorithm could be used to solve for any elastic tensor. This exposed me to various new aspects of computational research, from the Linux operating system and shell scripting to Fortran, and resulted in my senior thesis.\\

Now beginning my second year of graduate school, outside of completing coursework, my time has been spent investigating two stochastic networks. The first is a variant of a voting model which details, at a very basic level, the flow of opinions through a society. The initial state of the system is a certain random network with $n$ nodes, and each node is assigned one of $k$ opinions. As the system evolves, the nodes' opinions change, and connections, or edges, between the nodes are formed and broken.  Eventually, a final state is reached in which each edge connects nodes of the same opinion, i.e. if an edge $e_{ij}$ with ends at nodes $v_{i}$ and $v_{j}$ exists in the final state, the opinions of $v_{i}$ and $v_{j}$ will match. Through numerical simulations, the macroscopic variables governing the fine-scale system evolution were uncovered. These were then used in a coarse projective integration scheme to reduce simulation times by up to one third. Future work in this direction will involve using data-mining techniques to find these coarse variables, which will serve as a proof of concept for further work.\\

The details of the second system under investigation, an ``edge reconnecting'' model, are more involved. Namely, the system admits multiple connections between two nodes, complicating the underlying network description. Two distinct timescales arise from the dynamics, $T\asymp n^{2}$ and $T\asymp n^{3}$ where $T$ is the number of simulation steps taken. On the faster, $O(n^{2})$ scale, the degrees of the vertices may be considered constant, while the number of parallel edges between vertices changes. On the slower $O(n^{3})$ scale, the degree distribution evolves to a steady state value. This knowledge was used to again implement a coarse projective integrator, the results of which are still being tested. The macroscopic variables we've used here involve are complicated and not readily observed, being based on the spectrum of the network's adjacency matrix. As coarse descriptions aren't unique, we plan to bring dimensionality reduciont methods to bear on the system to search for simpler low-dimensional variables.\\  

Both of these investigations have been performed largely independently. My advisor, Professor Kevrekidis, is always willing to answer questions that may arise, but he largely leaves the details up to the individual student. I appreciate this approach, as it forces me to more actively think through problems and their potential solutions. It will certainly equip me with the ability to perform independent work in the future.\\

I very much enjoy life as a graduate student. As stressful, frustrating and hectic as it may be at times. I do not know why I'm not nervously listening for airstrikes overhead in Syria or wondering where my next meal will come from in a New Delhi slum, and probably never will, but I ought not take what I have been given for granted. What am I doing? Plotting a course into the future. Investigating real-world applicaitons for our research, from food supply to epidimiology. All the while establishing the foundation of knowledge necessary to carry out this meaningful work. Perhaps you will dismiss this as the naive outlook of a twenty-two-year-old, but these are my sincere desires. Empower me with the tools to pursue them.

\end{document}
